% !TeX root = RJwrapper.tex
\title{Lending Environment Simulator \& Lender Evaluation Tool}
\author{by Vadim Spirkov, Murlidhar Loka}

\maketitle

\abstract{%
In
}

% Any extra LaTeX you need in the preamble

\hypertarget{background}{%
\section{Background}\label{background}}

\hypertarget{objectives}{%
\section{Objectives}\label{objectives}}

\hypertarget{project-artifacts}{%
\section{Project Artifacts}\label{project-artifacts}}

\hypertarget{data-analysis}{%
\section{Data Analysis}\label{data-analysis}}

As it has been mentioned KASI Insight does not posses personal financial
data that could be used to identify and predict client credit
worthiness. In this project we are using the survey data collected by
KASI Insight over years from seven African countries. The survey targets
people who lend money on regular basis. The survey contains questions
pertaining to the borrowing habits of the people the lenders deal with,
asks respondents what they feel about the economy in a given country,
etc. KASI insight has collected almost \textbf{30,000} records over a
course of last three years.

\hypertarget{data-dictionary}{%
\subsection{Data Dictionary}\label{data-dictionary}}

The survey comprises 38 columns. Majority of them are multi-choice
questions. The table below lists the survey columns

\begin{longtable}[]{@{}ll@{}}
\toprule
\begin{minipage}[b]{0.05\columnwidth}\raggedright
ID\strut
\end{minipage} & \begin{minipage}[b]{0.89\columnwidth}\raggedright
Question/Column\strut
\end{minipage}\tabularnewline
\midrule
\endhead
\begin{minipage}[t]{0.05\columnwidth}\raggedright
0\strut
\end{minipage} & \begin{minipage}[t]{0.89\columnwidth}\raggedright
Timestamp\strut
\end{minipage}\tabularnewline
\begin{minipage}[t]{0.05\columnwidth}\raggedright
1\strut
\end{minipage} & \begin{minipage}[t]{0.89\columnwidth}\raggedright
Location ID\strut
\end{minipage}\tabularnewline
\begin{minipage}[t]{0.05\columnwidth}\raggedright
2\strut
\end{minipage} & \begin{minipage}[t]{0.89\columnwidth}\raggedright
Has it become more difficult or easier to find a job in your city?\strut
\end{minipage}\tabularnewline
\begin{minipage}[t]{0.05\columnwidth}\raggedright
3\strut
\end{minipage} & \begin{minipage}[t]{0.89\columnwidth}\raggedright
Is this a good time for people to make a large purchase such as
furniture or electrical appliances, given the economic climate?\strut
\end{minipage}\tabularnewline
\begin{minipage}[t]{0.05\columnwidth}\raggedright
4\strut
\end{minipage} & \begin{minipage}[t]{0.89\columnwidth}\raggedright
Compared to the last 6 months, are you able to spend (more, the same or
less) money on large purchases over the next 6 months?\strut
\end{minipage}\tabularnewline
\begin{minipage}[t]{0.05\columnwidth}\raggedright
5\strut
\end{minipage} & \begin{minipage}[t]{0.89\columnwidth}\raggedright
Will you be able to meet your regular expenses over the next 6
months?\strut
\end{minipage}\tabularnewline
\begin{minipage}[t]{0.05\columnwidth}\raggedright
6\strut
\end{minipage} & \begin{minipage}[t]{0.89\columnwidth}\raggedright
How do you expect your household's income to change over the next 6
months?\strut
\end{minipage}\tabularnewline
\begin{minipage}[t]{0.05\columnwidth}\raggedright
7\strut
\end{minipage} & \begin{minipage}[t]{0.89\columnwidth}\raggedright
How do you expect general economic conditions in your city to change
over the next 6 months?\strut
\end{minipage}\tabularnewline
\begin{minipage}[t]{0.05\columnwidth}\raggedright
8\strut
\end{minipage} & \begin{minipage}[t]{0.89\columnwidth}\raggedright
How do you expect general economic conditions in your country to change
over the next 6 months?\strut
\end{minipage}\tabularnewline
\begin{minipage}[t]{0.05\columnwidth}\raggedright
9\strut
\end{minipage} & \begin{minipage}[t]{0.89\columnwidth}\raggedright
Gender\strut
\end{minipage}\tabularnewline
\begin{minipage}[t]{0.05\columnwidth}\raggedright
10\strut
\end{minipage} & \begin{minipage}[t]{0.89\columnwidth}\raggedright
Marital status\strut
\end{minipage}\tabularnewline
\begin{minipage}[t]{0.05\columnwidth}\raggedright
11\strut
\end{minipage} & \begin{minipage}[t]{0.89\columnwidth}\raggedright
Age\strut
\end{minipage}\tabularnewline
\begin{minipage}[t]{0.05\columnwidth}\raggedright
12\strut
\end{minipage} & \begin{minipage}[t]{0.89\columnwidth}\raggedright
What's your highest level of education?\strut
\end{minipage}\tabularnewline
\begin{minipage}[t]{0.05\columnwidth}\raggedright
13\strut
\end{minipage} & \begin{minipage}[t]{0.89\columnwidth}\raggedright
Occupation\strut
\end{minipage}\tabularnewline
\begin{minipage}[t]{0.05\columnwidth}\raggedright
14\strut
\end{minipage} & \begin{minipage}[t]{0.89\columnwidth}\raggedright
If you are a student, what level are you currently studying?\strut
\end{minipage}\tabularnewline
\begin{minipage}[t]{0.05\columnwidth}\raggedright
15\strut
\end{minipage} & \begin{minipage}[t]{0.89\columnwidth}\raggedright
Race/Ethnicity\strut
\end{minipage}\tabularnewline
\begin{minipage}[t]{0.05\columnwidth}\raggedright
16\strut
\end{minipage} & \begin{minipage}[t]{0.89\columnwidth}\raggedright
Country\strut
\end{minipage}\tabularnewline
\begin{minipage}[t]{0.05\columnwidth}\raggedright
17\strut
\end{minipage} & \begin{minipage}[t]{0.89\columnwidth}\raggedright
What is the name of the neighborhood where you live?\strut
\end{minipage}\tabularnewline
\begin{minipage}[t]{0.05\columnwidth}\raggedright
18\strut
\end{minipage} & \begin{minipage}[t]{0.89\columnwidth}\raggedright
Over the past 3 months, how many times have you lent someone
money?\strut
\end{minipage}\tabularnewline
\begin{minipage}[t]{0.05\columnwidth}\raggedright
19\strut
\end{minipage} & \begin{minipage}[t]{0.89\columnwidth}\raggedright
On average how much do you lend in general?\strut
\end{minipage}\tabularnewline
\begin{minipage}[t]{0.05\columnwidth}\raggedright
20\strut
\end{minipage} & \begin{minipage}[t]{0.89\columnwidth}\raggedright
Who did you lend money to in the past 3 months?\strut
\end{minipage}\tabularnewline
\begin{minipage}[t]{0.05\columnwidth}\raggedright
21\strut
\end{minipage} & \begin{minipage}[t]{0.89\columnwidth}\raggedright
When you lend money, when do you usually expect to get it repaid?\strut
\end{minipage}\tabularnewline
\begin{minipage}[t]{0.05\columnwidth}\raggedright
22\strut
\end{minipage} & \begin{minipage}[t]{0.89\columnwidth}\raggedright
Do you include either interest or a lending fee when you lend?\strut
\end{minipage}\tabularnewline
\begin{minipage}[t]{0.05\columnwidth}\raggedright
23\strut
\end{minipage} & \begin{minipage}[t]{0.89\columnwidth}\raggedright
Do you request guarantees when you lend?\strut
\end{minipage}\tabularnewline
\begin{minipage}[t]{0.05\columnwidth}\raggedright
24\strut
\end{minipage} & \begin{minipage}[t]{0.89\columnwidth}\raggedright
Do you receive your money back in time?\strut
\end{minipage}\tabularnewline
\begin{minipage}[t]{0.05\columnwidth}\raggedright
25\strut
\end{minipage} & \begin{minipage}[t]{0.89\columnwidth}\raggedright
Assuming that you have lent money at least ten times, how often would
you get your money repaid?\strut
\end{minipage}\tabularnewline
\begin{minipage}[t]{0.05\columnwidth}\raggedright
26\strut
\end{minipage} & \begin{minipage}[t]{0.89\columnwidth}\raggedright
What's the most common use of the money you lend?\strut
\end{minipage}\tabularnewline
\begin{minipage}[t]{0.05\columnwidth}\raggedright
27\strut
\end{minipage} & \begin{minipage}[t]{0.89\columnwidth}\raggedright
Have you ever applied for a bank loan?\strut
\end{minipage}\tabularnewline
\begin{minipage}[t]{0.05\columnwidth}\raggedright
28\strut
\end{minipage} & \begin{minipage}[t]{0.89\columnwidth}\raggedright
Are you a tontine / lending club member?\strut
\end{minipage}\tabularnewline
\begin{minipage}[t]{0.05\columnwidth}\raggedright
29\strut
\end{minipage} & \begin{minipage}[t]{0.89\columnwidth}\raggedright
What is the most convenient way to get a loan?\strut
\end{minipage}\tabularnewline
\begin{minipage}[t]{0.05\columnwidth}\raggedright
30\strut
\end{minipage} & \begin{minipage}[t]{0.89\columnwidth}\raggedright
To what extent do you agree with the following sentences {[}Access to
credit is essential for me to achieve financial freedom{]}\strut
\end{minipage}\tabularnewline
\begin{minipage}[t]{0.05\columnwidth}\raggedright
31\strut
\end{minipage} & \begin{minipage}[t]{0.89\columnwidth}\raggedright
To what extent do you agree with the following sentences {[}Credit is
beneficial only if you have discipline{]}\strut
\end{minipage}\tabularnewline
\begin{minipage}[t]{0.05\columnwidth}\raggedright
32\strut
\end{minipage} & \begin{minipage}[t]{0.89\columnwidth}\raggedright
To what extent do you agree with the following sentences {[}I would like
to have more credit management training{]}\strut
\end{minipage}\tabularnewline
\begin{minipage}[t]{0.05\columnwidth}\raggedright
33\strut
\end{minipage} & \begin{minipage}[t]{0.89\columnwidth}\raggedright
What type of loans are you currently paying of?\strut
\end{minipage}\tabularnewline
\begin{minipage}[t]{0.05\columnwidth}\raggedright
34\strut
\end{minipage} & \begin{minipage}[t]{0.89\columnwidth}\raggedright
Do you have a credit score?\strut
\end{minipage}\tabularnewline
\begin{minipage}[t]{0.05\columnwidth}\raggedright
35\strut
\end{minipage} & \begin{minipage}[t]{0.89\columnwidth}\raggedright
Do you have a credit card?\strut
\end{minipage}\tabularnewline
\begin{minipage}[t]{0.05\columnwidth}\raggedright
36\strut
\end{minipage} & \begin{minipage}[t]{0.89\columnwidth}\raggedright
On average, how much of your total household monthly income do you spend
paying off debt each month?\strut
\end{minipage}\tabularnewline
\begin{minipage}[t]{0.05\columnwidth}\raggedright
37\strut
\end{minipage} & \begin{minipage}[t]{0.89\columnwidth}\raggedright
If you wanted to take a loan to start a business, how much would you
need?\strut
\end{minipage}\tabularnewline
\bottomrule
\end{longtable}

The survey data could be split into three major categories:

\begin{itemize}
\tightlist
\item
  Demographic Statistics
\item
  Economic Sentiment
\item
  Spending and Borrowing habits
\end{itemize}

Each question/column should be treated as a categorical value.

\hypertarget{data-exploration}{%
\subsection{Data Exploration}\label{data-exploration}}

Let's take a look at the raw survey data. Though the survey multi-choice
questions are categories in nature, the survey answers are stored in
alphanumeric format. Overtime some questions have been rephrased. Thus
in some cases the answers that pertain to the same category vary.
Another problem with the raw data set is the missing values.

\textbf{TO DO: insert the sample of raw data here}

To rectify the problems stated above we have developed a data processing
algorithm that normalized and categorized the answers converting them
into numeric form. The data processing script also imputes the missing
data with the most frequently occurring value for a given category. The
clean data set stats are depicted below. \textbf{Note}: the column
numbers correspond to the question numbers as described in \emph{Data
Dictionary} paragraph.

\begin{Schunk}
\begin{Soutput}
#>               2         3         4         5         6         7         8         9        10  \
#> count 29,383.00 29,383.00 29,383.00 29,383.00 29,383.00 29,383.00 29,383.00 29,383.00 29,383.00   
#> mean       1.69      2.07      1.90      2.16      1.80      1.82      1.83      1.66      1.99   
#> std        0.62      0.61      0.64      0.60      0.64      0.63      0.64      0.48      0.87   
#> min        1.00      1.00      1.00      1.00      1.00      1.00      1.00      1.00      1.00   
#> 25%        1.00      2.00      1.00      2.00      1.00      1.00      1.00      1.00      1.00   
#> 50%        2.00      2.00      2.00      2.00      2.00      2.00      2.00      2.00      2.00   
#> 75%        2.00      2.00      2.00      3.00      2.00      2.00      2.00      2.00      2.00   
#> max        3.00      3.00      3.00      3.00      3.00      3.00      3.00      2.00      7.00   
#> 
#>              11        12        13        14        16        18        19        20        21  \
#> count 29,383.00 29,383.00 29,383.00 29,383.00 29,383.00 29,383.00 29,383.00 29,383.00 29,383.00   
#> mean       3.68      3.85      2.77      0.24      3.59      2.41      2.07      1.93      2.39   
#> std        1.17      1.27      1.20      0.71      1.93      0.90      0.75      0.61      0.97   
#> min        1.00      1.00      1.00      0.00      1.00      1.00      1.00      1.00      1.00   
#> 25%        3.00      3.00      2.00      0.00      2.00      2.00      2.00      2.00      2.00   
#> 50%        4.00      4.00      3.00      0.00      4.00      2.00      2.00      2.00      2.00   
#> 75%        4.00      5.00      4.00      0.00      5.00      3.00      2.00      2.00      3.00   
#> max        9.00     10.00      9.00      4.00      7.00      4.00      4.00      3.00      6.00   
#> 
#>              22        23        24        25        26        27        28        29        30  \
#> count 29,383.00 29,383.00 29,383.00 29,383.00 29,383.00 29,383.00 29,383.00 29,383.00 29,383.00   
#> mean       3.26      3.28      2.80      2.71      2.61      1.47      1.51      2.66      2.71   
#> std        1.32      1.32      1.16      1.10      1.38      0.50      0.50      1.06      1.10   
#> min        1.00      1.00      1.00      1.00      1.00      1.00      1.00      1.00      1.00   
#> 25%        2.00      2.00      2.00      2.00      1.00      1.00      1.00      2.00      2.00   
#> 50%        3.00      3.00      3.00      3.00      2.00      1.00      2.00      2.00      2.00   
#> 75%        5.00      5.00      4.00      3.00      3.00      2.00      2.00      3.00      3.00   
#> max        5.00      5.00      5.00      5.00      5.00      2.00      2.00      6.00      5.00   
#> 
#>              31        32        33        34        35        36        37  credit_score  \
#> count 29,383.00 29,383.00 29,383.00 29,383.00 29,383.00 29,383.00 29,383.00     29,383.00   
#> mean       2.84      2.90      0.30      1.25      1.30      7.68      2.19        348.56   
#> std        1.13      1.22      1.08      0.44      0.46      1.26      0.58        244.70   
#> min        1.00      1.00      0.00      1.00      1.00      1.00      1.00       -720.00   
#> 25%        2.00      2.00      0.00      1.00      1.00      8.00      2.00        190.00   
#> 50%        2.00      2.00      0.00      1.00      1.00      8.00      2.00        390.00   
#> 75%        4.00      4.00      0.00      2.00      2.00      8.00      2.00        540.00   
#> max        5.00      5.00      8.00      2.00      2.00      8.00      4.00        970.00   
#> 
#>        credit_score_category  lender_score  lender_score_category  
#> count              29,383.00     29,383.00              29,383.00  
#> mean                    1.84        303.80                   1.74  
#> std                     0.95        272.73                   0.82  
#> min                     1.00       -590.00                   1.00  
#> 25%                     1.00        140.00                   1.00  
#> 50%                     2.00        380.00                   2.00  
#> 75%                     2.00        500.00                   2.00  
#> max                     5.00        840.00                   5.00
\end{Soutput}
\end{Schunk}

Evidently now all the data is categorized, the missing values imputed.
From this point on we will be using the clean data set to do further
data exploration, feature engineering and model training.

\hypertarget{demographic-stats}{%
\subsubsection{Demographic Stats}\label{demographic-stats}}

It is useful to understand who took the survey. This knowledge will
ultimately give us the answers about the money market participants in
Africa.

\begin{Schunk}
\begin{figure}[H]

{\centering \includegraphics[width=1.15\linewidth]{../../artifacts/obs_per_country} 

}

\caption[Participation per Country]{Participation per Country}\label{fig:pc}
\end{figure}
\end{Schunk}

\begin{Schunk}
\begin{figure}[H]

{\centering \includegraphics[width=1.15\linewidth]{../../artifacts/participant_age} 

}

\caption[Age of Participans]{Age of Participans}\label{fig:pa}
\end{figure}
\end{Schunk}

\begin{Schunk}
\begin{figure}[H]

{\centering \includegraphics[width=1.15\linewidth]{../../artifacts/participant_education} 

}

\caption[Education of Participants]{Education of Participants}\label{fig:pe}
\end{figure}
\end{Schunk}

\begin{Schunk}
\begin{figure}[H]

{\centering \includegraphics[width=1.15\linewidth]{../../artifacts/participant_marital} 

}

\caption[Marital Status of Participants]{Marital Status of Participants}\label{fig:pms}
\end{figure}
\end{Schunk}

As per the charts submitted above we can conclude that:

\begin{itemize}
\tightlist
\item
  Cameroon has the highest number of observations and Tanzania has the
  smallest representation, where the rest of the countries or more or
  less equally represented.
\item
  Males dominate in the money lending business. Kenya though makes an
  exception where number of female participants is very close to the
  male population
\item
  In general people in \emph{30-34} age group are the most active,
  followed by \emph{25-29} and \emph{18-24} age groups respectively. In
  Kenya, unlike other countries, the younger generation is more active.
\item
  Majority of money lenders are either salaried or commission-based
  employees. Again Kenya makes an exception. The second largest group of
  the money lenders is the business owners.
\item
  Education-wise people with the bachelor's degree and skilled trade
  workers dominate.
\item
  Married people tend to lend money more often\ldots{}
\end{itemize}

\hypertarget{economic-sentiment}{%
\subsubsection{Economic Sentiment}\label{economic-sentiment}}

Now let's see what the money lenders think about the state of the
economy in their respective countries. The questions where asked in the
six-month perspective in the future from the date of survey.

\begin{Schunk}
\begin{figure}[H]

{\centering \includegraphics[width=1.15\linewidth]{../../artifacts/sentiment} 

}

\caption[Economic Sentiment]{Economic Sentiment}\label{fig:esent}
\end{figure}
\end{Schunk}

Evidently majority of the survey participants think that the economic
situation in their country will be stable over a course of next six
months. Many people in Kenya, Nigeria and Ghana find it more difficult
to find a job. Remarkably, despite the fact that people believe that the
economic conditions are stable, citizens of all counties are not sure if
they are going to meet their regular expenses.

\hypertarget{spending-and-borrowing-habits}{%
\subsubsection{Spending and Borrowing
Habits}\label{spending-and-borrowing-habits}}

Spending and borrowing habits is the segment of our particular interest
since it affects the most the credit score of the population.

\begin{Schunk}
\begin{figure}[H]

{\centering \includegraphics[width=1.15\linewidth]{../../artifacts/borrowing} 

}

\caption[Borrowing Habits]{Borrowing Habits}\label{fig:bh}
\end{figure}
\end{Schunk}

\begin{Schunk}
\begin{figure}[H]

{\centering \includegraphics[width=1.15\linewidth]{../../artifacts/spending} 

}

\caption[Spending Habits]{Spending Habits}\label{fig:sh}
\end{figure}
\end{Schunk}

\begin{Schunk}
\begin{figure}[H]

{\centering \includegraphics[width=1.15\linewidth]{../../artifacts/payment} 

}

\caption[Debt Payment]{Debt Payment}\label{fig:dtp}
\end{figure}
\end{Schunk}

\begin{itemize}
\tightlist
\item
  Majority of population take either small or micro loans (the exact
  amounts are country specific).
\item
  It is quite remarkable that the lenders do not charge fees or interest
  regularly (if at all) more often than not. The Cameroonians make an
  exception. In opposite the majority of South African lenders never
  change interest. We have conducted further data research that have
  proved that many people tend to lend to friends and family. This fact
  explains why the fees and interest on loans are waived.
\item
  People in Cameroon, Cote d'Ivoire and Tanzania spend the loans to
  cover business-related expenses. Citizens of other countries mainly
  use loan to either cover one-time or or unexpected expenses (wedding,
  medical emergency..) or make ends meet (pay rent, buy clothes, etc.)
\item
  Interestingly people in all countries do not watch how they spend the
  borrowed money. This fact probably explains why the question
  \emph{Will you be able to meet your regular expenses?} generates
  uncertain answers (see \textbf{Economic Sentiment} paragraph for
  further details).
\end{itemize}

\hypertarget{data-distribution-between-categories}{%
\subsection{Data Distribution between
Categories}\label{data-distribution-between-categories}}

There are five credit categories for borrowers and five lender
categories. To train the robust classification models we have to ensure
that each category has enough observations to support the model
training. Let's review the data distribution between the borrower and
lender categories.

\begin{Schunk}
\begin{figure}[H]

{\centering \includegraphics[width=1.15\linewidth]{../../artifacts/bcategories} 

}

\caption[Data Distribution per Credit Categories]{Data Distribution per Credit Categories}\label{fig:dc}
\end{figure}
\end{Schunk}

\begin{Schunk}
\begin{figure}[H]

{\centering \includegraphics[width=1.15\linewidth]{../../artifacts/lcategories} 

}

\caption[Data Distribution per Lender Categories]{Data Distribution per Lender Categories}\label{fig:dl}
\end{figure}
\end{Schunk}

As we can observe overall the lending environment is not very promising;
categories 1 and 2 (\emph{Very Poor} and \emph{Poor}) dominate. The
lending climate is visibly better in Cameroon, Cote d'Ivoire and South
Africa. It is also worth mentioning that categories 4 and 5 (\emph{Good}
and \emph{Very Good}) do not have that much data. The situation is even
worse with the lender categories. Thus prior to the model training we
would have to upsample the training data sets to bring all categories to
the same level.

Overall looking at the credit and lender scores of the population we
observe that the distribution pattern is very similar between all seven
African countries. Thus if KASI Insight adds more countries to the fold
there is no need to retrain the models assuming that the newly added
countries have the same category distribution\ldots{}

\hypertarget{feature-selection-and-engineering}{%
\section{Feature Selection and
Engineering}\label{feature-selection-and-engineering}}

The data set has 38 columns. We potentially, could employ all of them to
fit the models. But this is not the optimal approach. Not all data
elements contribute to the category identification equally, some may not
contribute at all, so why keep them? Another consideration is that the
large and wide data sets make model training much longer, affect the
accuracy and speed of the models negatively. Also many input variables
add complexity to the user interface making it hard to implement,
maintain and use. Thus we have opted to evaluated available data
features. The ultimate goal is to understand the relationship between
the features and the response variables and select the most influential
ones.

\hypertarget{feature-correlation-matrix}{%
\subsection{Feature Correlation
Matrix}\label{feature-correlation-matrix}}

Strongly correlated features are redundant thus they could be dropped
without impacting the model performance. Figure \ref{fig:cmatrix}
depicts a correlation heatmap of all 38 data set features. The
correlated features would be rendered either in deep black or very light
colors. As we can observe none of the features have strong correlation.

\begin{Schunk}
\begin{figure}[H]

{\centering \includegraphics[width=0.75\linewidth]{../../artifacts/cmatrix} 

}

\caption[Feature Correlation]{Feature Correlation}\label{fig:cmatrix}
\end{figure}
\end{Schunk}

\hypertarget{univariate-feature-selection}{%
\subsection{Univariate Feature
Selection}\label{univariate-feature-selection}}

Univariate feature selection examines each feature individually to
determine the strength of the relationship of the feature with the
response variable. Next two paragraphs examine relationship between top
20 features and credit and lender categories respectively.

\hypertarget{credit-score-univariate-feature-selection}{%
\subsubsection{Credit Score Univariate Feature
Selection}\label{credit-score-univariate-feature-selection}}

\begin{longtable}[]{@{}lll@{}}
\toprule
\begin{minipage}[b]{0.05\columnwidth}\raggedright
Num\strut
\end{minipage} & \begin{minipage}[b]{0.77\columnwidth}\raggedright
Feature\strut
\end{minipage} & \begin{minipage}[b]{0.09\columnwidth}\raggedright
Score\strut
\end{minipage}\tabularnewline
\midrule
\endhead
\begin{minipage}[t]{0.05\columnwidth}\raggedright
24\strut
\end{minipage} & \begin{minipage}[t]{0.77\columnwidth}\raggedright
Do you receive your money back in time?\strut
\end{minipage} & \begin{minipage}[t]{0.09\columnwidth}\raggedright
4749.1\strut
\end{minipage}\tabularnewline
\begin{minipage}[t]{0.05\columnwidth}\raggedright
18\strut
\end{minipage} & \begin{minipage}[t]{0.77\columnwidth}\raggedright
Over the past 3 months, how many times have you lent someone
money?\strut
\end{minipage} & \begin{minipage}[t]{0.09\columnwidth}\raggedright
1077.69\strut
\end{minipage}\tabularnewline
\begin{minipage}[t]{0.05\columnwidth}\raggedright
26\strut
\end{minipage} & \begin{minipage}[t]{0.77\columnwidth}\raggedright
What's the most common use of the money you lend?\strut
\end{minipage} & \begin{minipage}[t]{0.09\columnwidth}\raggedright
727.16\strut
\end{minipage}\tabularnewline
\begin{minipage}[t]{0.05\columnwidth}\raggedright
22\strut
\end{minipage} & \begin{minipage}[t]{0.77\columnwidth}\raggedright
Do you include either interest or a lending fee when you lend?\strut
\end{minipage} & \begin{minipage}[t]{0.09\columnwidth}\raggedright
536.27\strut
\end{minipage}\tabularnewline
\begin{minipage}[t]{0.05\columnwidth}\raggedright
19\strut
\end{minipage} & \begin{minipage}[t]{0.77\columnwidth}\raggedright
On average how much do you lend in general?\strut
\end{minipage} & \begin{minipage}[t]{0.09\columnwidth}\raggedright
512.08\strut
\end{minipage}\tabularnewline
\begin{minipage}[t]{0.05\columnwidth}\raggedright
20\strut
\end{minipage} & \begin{minipage}[t]{0.77\columnwidth}\raggedright
Who did you lend money to in the past 3 months?\strut
\end{minipage} & \begin{minipage}[t]{0.09\columnwidth}\raggedright
441.73\strut
\end{minipage}\tabularnewline
\begin{minipage}[t]{0.05\columnwidth}\raggedright
33\strut
\end{minipage} & \begin{minipage}[t]{0.77\columnwidth}\raggedright
What type of loans are you currently paying of?\strut
\end{minipage} & \begin{minipage}[t]{0.09\columnwidth}\raggedright
360.28\strut
\end{minipage}\tabularnewline
\begin{minipage}[t]{0.05\columnwidth}\raggedright
23\strut
\end{minipage} & \begin{minipage}[t]{0.77\columnwidth}\raggedright
Do you request guarantees when you lend?\strut
\end{minipage} & \begin{minipage}[t]{0.09\columnwidth}\raggedright
351.02\strut
\end{minipage}\tabularnewline
\begin{minipage}[t]{0.05\columnwidth}\raggedright
14\strut
\end{minipage} & \begin{minipage}[t]{0.77\columnwidth}\raggedright
If you are a student, what level are you currently studying?\strut
\end{minipage} & \begin{minipage}[t]{0.09\columnwidth}\raggedright
301.56\strut
\end{minipage}\tabularnewline
\begin{minipage}[t]{0.05\columnwidth}\raggedright
21\strut
\end{minipage} & \begin{minipage}[t]{0.77\columnwidth}\raggedright
When you lend money, when do you usually expect to get it repaid?\strut
\end{minipage} & \begin{minipage}[t]{0.09\columnwidth}\raggedright
218.30\strut
\end{minipage}\tabularnewline
\begin{minipage}[t]{0.05\columnwidth}\raggedright
16\strut
\end{minipage} & \begin{minipage}[t]{0.77\columnwidth}\raggedright
Country\strut
\end{minipage} & \begin{minipage}[t]{0.09\columnwidth}\raggedright
197.034\strut
\end{minipage}\tabularnewline
\begin{minipage}[t]{0.05\columnwidth}\raggedright
25\strut
\end{minipage} & \begin{minipage}[t]{0.77\columnwidth}\raggedright
Assuming that you have lent money at least ten times, how often would
you get your money repaid?\strut
\end{minipage} & \begin{minipage}[t]{0.09\columnwidth}\raggedright
180.84\strut
\end{minipage}\tabularnewline
\begin{minipage}[t]{0.05\columnwidth}\raggedright
11\strut
\end{minipage} & \begin{minipage}[t]{0.77\columnwidth}\raggedright
Age\strut
\end{minipage} & \begin{minipage}[t]{0.09\columnwidth}\raggedright
103.30\strut
\end{minipage}\tabularnewline
\begin{minipage}[t]{0.05\columnwidth}\raggedright
12\strut
\end{minipage} & \begin{minipage}[t]{0.77\columnwidth}\raggedright
What's your highest level of education?\strut
\end{minipage} & \begin{minipage}[t]{0.09\columnwidth}\raggedright
75.24\strut
\end{minipage}\tabularnewline
\begin{minipage}[t]{0.05\columnwidth}\raggedright
29\strut
\end{minipage} & \begin{minipage}[t]{0.77\columnwidth}\raggedright
What is the most convenient way to get a loan?\strut
\end{minipage} & \begin{minipage}[t]{0.09\columnwidth}\raggedright
63.37\strut
\end{minipage}\tabularnewline
\begin{minipage}[t]{0.05\columnwidth}\raggedright
2\strut
\end{minipage} & \begin{minipage}[t]{0.77\columnwidth}\raggedright
Has it become more difficult or easier to find a job in your city?\strut
\end{minipage} & \begin{minipage}[t]{0.09\columnwidth}\raggedright
55.15\strut
\end{minipage}\tabularnewline
\begin{minipage}[t]{0.05\columnwidth}\raggedright
10\strut
\end{minipage} & \begin{minipage}[t]{0.77\columnwidth}\raggedright
Marital status\strut
\end{minipage} & \begin{minipage}[t]{0.09\columnwidth}\raggedright
39.54\strut
\end{minipage}\tabularnewline
\begin{minipage}[t]{0.05\columnwidth}\raggedright
31\strut
\end{minipage} & \begin{minipage}[t]{0.77\columnwidth}\raggedright
To what extent do you agree with the following sentences {[}Credit is
beneficial only if you h\ldots{}\strut
\end{minipage} & \begin{minipage}[t]{0.09\columnwidth}\raggedright
35.68\strut
\end{minipage}\tabularnewline
\begin{minipage}[t]{0.05\columnwidth}\raggedright
30\strut
\end{minipage} & \begin{minipage}[t]{0.77\columnwidth}\raggedright
To what extent do you agree with the following sentences {[}Access to
credit is essential for \ldots{}\strut
\end{minipage} & \begin{minipage}[t]{0.09\columnwidth}\raggedright
31.48\strut
\end{minipage}\tabularnewline
\begin{minipage}[t]{0.05\columnwidth}\raggedright
32\strut
\end{minipage} & \begin{minipage}[t]{0.77\columnwidth}\raggedright
To what extent do you agree with the following sentences {[}I would like
to have more credit m\ldots{}\strut
\end{minipage} & \begin{minipage}[t]{0.09\columnwidth}\raggedright
25.52\strut
\end{minipage}\tabularnewline
\bottomrule
\end{longtable}

\hypertarget{lender-score-univariate-feature-selection}{%
\subsubsection{Lender Score Univariate Feature
Selection}\label{lender-score-univariate-feature-selection}}

\begin{longtable}[]{@{}lll@{}}
\toprule
\begin{minipage}[b]{0.05\columnwidth}\raggedright
Num\strut
\end{minipage} & \begin{minipage}[b]{0.77\columnwidth}\raggedright
Feature\strut
\end{minipage} & \begin{minipage}[b]{0.09\columnwidth}\raggedright
Score\strut
\end{minipage}\tabularnewline
\midrule
\endhead
\begin{minipage}[t]{0.05\columnwidth}\raggedright
24\strut
\end{minipage} & \begin{minipage}[t]{0.77\columnwidth}\raggedright
Do you receive your money back in time?\strut
\end{minipage} & \begin{minipage}[t]{0.09\columnwidth}\raggedright
6667.83\strut
\end{minipage}\tabularnewline
\begin{minipage}[t]{0.05\columnwidth}\raggedright
22\strut
\end{minipage} & \begin{minipage}[t]{0.77\columnwidth}\raggedright
Do you include either interest or a lending fee when you lend?\strut
\end{minipage} & \begin{minipage}[t]{0.09\columnwidth}\raggedright
3588.47\strut
\end{minipage}\tabularnewline
\begin{minipage}[t]{0.05\columnwidth}\raggedright
23\strut
\end{minipage} & \begin{minipage}[t]{0.77\columnwidth}\raggedright
Do you request guarantees when you lend?\strut
\end{minipage} & \begin{minipage}[t]{0.09\columnwidth}\raggedright
3339.37\strut
\end{minipage}\tabularnewline
\begin{minipage}[t]{0.05\columnwidth}\raggedright
18\strut
\end{minipage} & \begin{minipage}[t]{0.77\columnwidth}\raggedright
Over the past 3 months, how many times have you lent someone
money?\strut
\end{minipage} & \begin{minipage}[t]{0.09\columnwidth}\raggedright
2014.33\strut
\end{minipage}\tabularnewline
\begin{minipage}[t]{0.05\columnwidth}\raggedright
16\strut
\end{minipage} & \begin{minipage}[t]{0.77\columnwidth}\raggedright
Country\strut
\end{minipage} & \begin{minipage}[t]{0.09\columnwidth}\raggedright
595.87\strut
\end{minipage}\tabularnewline
\begin{minipage}[t]{0.05\columnwidth}\raggedright
25\strut
\end{minipage} & \begin{minipage}[t]{0.77\columnwidth}\raggedright
Assuming that you have lent money at least ten times, how often would
you get your money repaid?\strut
\end{minipage} & \begin{minipage}[t]{0.09\columnwidth}\raggedright
555.59\strut
\end{minipage}\tabularnewline
\begin{minipage}[t]{0.05\columnwidth}\raggedright
19\strut
\end{minipage} & \begin{minipage}[t]{0.77\columnwidth}\raggedright
On average how much do you lend in general?\strut
\end{minipage} & \begin{minipage}[t]{0.09\columnwidth}\raggedright
460.15\strut
\end{minipage}\tabularnewline
\begin{minipage}[t]{0.05\columnwidth}\raggedright
26\strut
\end{minipage} & \begin{minipage}[t]{0.77\columnwidth}\raggedright
What's the most common use of the money you lend?\strut
\end{minipage} & \begin{minipage}[t]{0.09\columnwidth}\raggedright
444.85\strut
\end{minipage}\tabularnewline
\begin{minipage}[t]{0.05\columnwidth}\raggedright
20\strut
\end{minipage} & \begin{minipage}[t]{0.77\columnwidth}\raggedright
Who did you lend money to in the past 3 months?\strut
\end{minipage} & \begin{minipage}[t]{0.09\columnwidth}\raggedright
415.17\strut
\end{minipage}\tabularnewline
\begin{minipage}[t]{0.05\columnwidth}\raggedright
14\strut
\end{minipage} & \begin{minipage}[t]{0.77\columnwidth}\raggedright
If you are a student, what level are you currently studying?\strut
\end{minipage} & \begin{minipage}[t]{0.09\columnwidth}\raggedright
174.17\strut
\end{minipage}\tabularnewline
\begin{minipage}[t]{0.05\columnwidth}\raggedright
21\strut
\end{minipage} & \begin{minipage}[t]{0.77\columnwidth}\raggedright
When you lend money, when do you usually expect to get it repaid?\strut
\end{minipage} & \begin{minipage}[t]{0.09\columnwidth}\raggedright
112.94\strut
\end{minipage}\tabularnewline
\begin{minipage}[t]{0.05\columnwidth}\raggedright
37\strut
\end{minipage} & \begin{minipage}[t]{0.77\columnwidth}\raggedright
If you wanted to take a loan to start a business, how much would you
need?\strut
\end{minipage} & \begin{minipage}[t]{0.09\columnwidth}\raggedright
91.9\strut
\end{minipage}\tabularnewline
\begin{minipage}[t]{0.05\columnwidth}\raggedright
28\strut
\end{minipage} & \begin{minipage}[t]{0.77\columnwidth}\raggedright
Are you a tontine / lending club member?\strut
\end{minipage} & \begin{minipage}[t]{0.09\columnwidth}\raggedright
78.77\strut
\end{minipage}\tabularnewline
\begin{minipage}[t]{0.05\columnwidth}\raggedright
27\strut
\end{minipage} & \begin{minipage}[t]{0.77\columnwidth}\raggedright
Have you ever applied for a bank loan?\strut
\end{minipage} & \begin{minipage}[t]{0.09\columnwidth}\raggedright
76.69\strut
\end{minipage}\tabularnewline
\begin{minipage}[t]{0.05\columnwidth}\raggedright
4\strut
\end{minipage} & \begin{minipage}[t]{0.77\columnwidth}\raggedright
Compared to the last 6 months, are you able to spend (more, the same or
less) money on large pur\ldots{}\strut
\end{minipage} & \begin{minipage}[t]{0.09\columnwidth}\raggedright
56.49\strut
\end{minipage}\tabularnewline
\begin{minipage}[t]{0.05\columnwidth}\raggedright
11\strut
\end{minipage} & \begin{minipage}[t]{0.77\columnwidth}\raggedright
Age\strut
\end{minipage} & \begin{minipage}[t]{0.09\columnwidth}\raggedright
44.33\strut
\end{minipage}\tabularnewline
\begin{minipage}[t]{0.05\columnwidth}\raggedright
8\strut
\end{minipage} & \begin{minipage}[t]{0.77\columnwidth}\raggedright
How do you expect general economic conditions in your country to change
over the next 6 months?\strut
\end{minipage} & \begin{minipage}[t]{0.09\columnwidth}\raggedright
42.94\strut
\end{minipage}\tabularnewline
\begin{minipage}[t]{0.05\columnwidth}\raggedright
32\strut
\end{minipage} & \begin{minipage}[t]{0.77\columnwidth}\raggedright
To what extent do you agree with the following sentences {[}I would like
to have more credit manag\ldots{}\strut
\end{minipage} & \begin{minipage}[t]{0.09\columnwidth}\raggedright
39.22\strut
\end{minipage}\tabularnewline
\begin{minipage}[t]{0.05\columnwidth}\raggedright
2\strut
\end{minipage} & \begin{minipage}[t]{0.77\columnwidth}\raggedright
Has it become more difficult or easier to find a job in your city?\strut
\end{minipage} & \begin{minipage}[t]{0.09\columnwidth}\raggedright
38.57\strut
\end{minipage}\tabularnewline
\begin{minipage}[t]{0.05\columnwidth}\raggedright
35\strut
\end{minipage} & \begin{minipage}[t]{0.77\columnwidth}\raggedright
Do you have a credit card?\strut
\end{minipage} & \begin{minipage}[t]{0.09\columnwidth}\raggedright
37.26\strut
\end{minipage}\tabularnewline
\bottomrule
\end{longtable}

\hypertarget{feature-importance}{%
\subsection{Feature Importance}\label{feature-importance}}

We measure the importance of a feature by calculating the increase in
the model's prediction error after permuting the feature. A feature is
``important'' if shuffling its values increases the model error, because
in this case the model relied on the feature for the prediction. A
feature is ``unimportant'' if shuffling its values leaves the model
error unchanged, because in this case the model ignored the feature for
the prediction.

\hypertarget{credit-score-feature-importance-evaluation}{%
\subsubsection{Credit Score Feature Importance
Evaluation}\label{credit-score-feature-importance-evaluation}}

\begin{Schunk}
\begin{figure}[H]

{\centering \includegraphics[width=1\linewidth]{../../artifacts/cfimportance} 

}

\caption[Credit Score Feature Impoirtance]{Credit Score Feature Impoirtance}\label{fig:cfi}
\end{figure}
\end{Schunk}

\hypertarget{lender-score-feature-importance-evaluation}{%
\subsubsection{Lender Score Feature Importance
Evaluation}\label{lender-score-feature-importance-evaluation}}

\begin{Schunk}
\begin{figure}[H]

{\centering \includegraphics[width=1\linewidth]{../../artifacts/lfimportance} 

}

\caption[Lender Score Feature Impoirtance]{Lender Score Feature Impoirtance}\label{fig:lfi}
\end{figure}
\end{Schunk}

\hypertarget{takeaways}{%
\subsection{Takeaways}\label{takeaways}}

We have applied two mathematical algorithms to identify the most
significant features for credit score and lender score labels. To no
surprise both methods have successfully identified the feature that have
been used to calculate the credit/ lender categories. We have decided to
select the top features that have distinctively higher score as a
base-line. During the model training and evaluation phase we will
increase/decrease the number of features to estimate the effect of the
input data dimentionality change the model accuracy.

\hypertarget{top-seven-credit-score-features}{%
\subsubsection{Top Seven Credit Score
Features}\label{top-seven-credit-score-features}}

\begin{longtable}[]{@{}ll@{}}
\toprule
Num & Feature\tabularnewline
\midrule
\endhead
24 & Do you receive your money back in time?\tabularnewline
26 & What's the most common use of the money you lend?\tabularnewline
22 & Do you include either interest or a lending fee when you
lend?\tabularnewline
18 & Over the past 3 months, how many times have you lent someone
money?\tabularnewline
20 & Who did you lend money to in the past 3 months?\tabularnewline
23 & Do you request guarantees when you lend?\tabularnewline
19 & On average how much do you lend in general?\tabularnewline
\bottomrule
\end{longtable}

\hypertarget{top-nine-lender-score-features}{%
\subsubsection{Top Nine Lender Score
Features}\label{top-nine-lender-score-features}}

\begin{longtable}[]{@{}ll@{}}
\toprule
\begin{minipage}[b]{0.05\columnwidth}\raggedright
Num\strut
\end{minipage} & \begin{minipage}[b]{0.89\columnwidth}\raggedright
Feature\strut
\end{minipage}\tabularnewline
\midrule
\endhead
\begin{minipage}[t]{0.05\columnwidth}\raggedright
24\strut
\end{minipage} & \begin{minipage}[t]{0.89\columnwidth}\raggedright
Do you receive your money back in time?\strut
\end{minipage}\tabularnewline
\begin{minipage}[t]{0.05\columnwidth}\raggedright
18\strut
\end{minipage} & \begin{minipage}[t]{0.89\columnwidth}\raggedright
Over the past 3 months, how many times have you lent someone
money?\strut
\end{minipage}\tabularnewline
\begin{minipage}[t]{0.05\columnwidth}\raggedright
22\strut
\end{minipage} & \begin{minipage}[t]{0.89\columnwidth}\raggedright
Do you include either interest or a lending fee when you lend?\strut
\end{minipage}\tabularnewline
\begin{minipage}[t]{0.05\columnwidth}\raggedright
23\strut
\end{minipage} & \begin{minipage}[t]{0.89\columnwidth}\raggedright
Do you request guarantees when you lend?\strut
\end{minipage}\tabularnewline
\begin{minipage}[t]{0.05\columnwidth}\raggedright
20\strut
\end{minipage} & \begin{minipage}[t]{0.89\columnwidth}\raggedright
Who did you lend money to in the past 3 months?\strut
\end{minipage}\tabularnewline
\begin{minipage}[t]{0.05\columnwidth}\raggedright
26\strut
\end{minipage} & \begin{minipage}[t]{0.89\columnwidth}\raggedright
What's the most common use of the money you lend?\strut
\end{minipage}\tabularnewline
\begin{minipage}[t]{0.05\columnwidth}\raggedright
19\strut
\end{minipage} & \begin{minipage}[t]{0.89\columnwidth}\raggedright
On average how much do you lend in general?\strut
\end{minipage}\tabularnewline
\begin{minipage}[t]{0.05\columnwidth}\raggedright
25\strut
\end{minipage} & \begin{minipage}[t]{0.89\columnwidth}\raggedright
Assuming that you have lent money at least ten times, how often would
you get your money repaid?\strut
\end{minipage}\tabularnewline
\begin{minipage}[t]{0.05\columnwidth}\raggedright
16\strut
\end{minipage} & \begin{minipage}[t]{0.89\columnwidth}\raggedright
Country\strut
\end{minipage}\tabularnewline
\bottomrule
\end{longtable}

\hypertarget{model-evaluation-and-selection}{%
\subsection{Model Evaluation and
Selection}\label{model-evaluation-and-selection}}

After we cleaned and normalized the data, labeled all observations and
gained deep understanding about the features we are ready to start model
training and evaluation. To gain the best result possible we will
explore and evaluate three algorithm to train the models. They are:

\begin{itemize}
\tightlist
\item
  \textbf{Support Vector Machine} (SVM). The greatest strength of SVM is
  that it has multiple Kernel implementations, that can be tuned to
  explain multi-dimensional space with high accuracy
\item
  \textbf{Random Forest} (RF). Random forest belongs to the class of
  ensemble models. It has many hyper-parameters that could be tuned to
  achieve high accuracy. The random forest algorithm is not demanding in
  terms of the data preparation, which makes it the first choice in many
  real-life scenarios
\item
  \textbf{Gradient Boosting Machine} (GBM). GBM is an ensemble model as
  well. It uses the concept of trees just like the RF model does but
  applies it differently. GBT builds the trees one at a time, where each
  new tree helps to correct errors made by previously trained tree. the
  GBM.
\end{itemize}

\hypertarget{evaluation-metrics}{%
\subsubsection{Evaluation Metrics}\label{evaluation-metrics}}

We believe that the best model has to classify all five categories as
accurate as possible. The winning model also would have to identify true
positives and true negatives for each category equally well. Thus we
choose the multiclass confusion matrix and F1 scores to evaluate the
models. The higher the F1 score for each category - the better the model
performs.

We also take into consideration the model training and inference speed.

\hypertarget{model-training-and-evaluation-methodology}{%
\subsubsection{Model Training and Evaluation
Methodology}\label{model-training-and-evaluation-methodology}}

\begin{itemize}
\tightlist
\item
  For the base-line models training we will use top seven feature for
  the simulator and top nine feature for the evaluator respectively.
\item
  We begin with the splitting the available data into the training
  (70\%) and test (30\%) sets.
\item
  We upsample the training data set employing \emph{SMOTE} algorithm.
\item
  We evaluate the three algorithms we have described above. We will be
  using the default algorithm parameters and top features (see
  \emph{Feature Evaluation} paragraph for more details) to fit the
  models.
\item
  We select the algorithm that has the best evaluation metrics.
\item
  Then we evaluate the winning algorithm fitting it with the smaller and
  larger feature sets.
\item
  If the data dimentionality change makes positive impact on the the
  winning algorithm we select this feature set for the model.
\item
  Finally we hyper-tune the algorithm parameters in effort to achieve
  even better model performance
\end{itemize}

\hypertarget{lending-environment-simulator-model}{%
\subsubsection{Lending Environment Simulator
Model}\label{lending-environment-simulator-model}}

Following the steps outlined in the previous section we have received
the following performances stats:

\hypertarget{svm}{%
\paragraph{SVM}\label{svm}}

\begin{verbatim}
              precision    recall  f1-score   support
       1       0.97      0.95      0.96      3935
       2       0.92      0.90      0.91      3114
       3       0.81      0.89      0.85      1122
       4       0.81      0.85      0.83       487
       5       0.81      0.89      0.85       157
       
micro avg      0.92      0.92      0.92      8815
macro avg      0.86      0.90      0.88      8815
weighted avg   0.92      0.92      0.92      8815

Overall algorithm accuracy: 0.9199
\end{verbatim}

\hypertarget{random-forest}{%
\paragraph{Random Forest}\label{random-forest}}

\begin{verbatim}
              precision    recall  f1-score   support
       1       0.98      0.96      0.97      3902
       2       0.93      0.93      0.93      3161
       3       0.86      0.90      0.88      1150
       4       0.86      0.84      0.85       464
       5       0.86      0.87      0.87       138
       
 micro avg     0.94      0.94      0.94      8815
 macro avg     0.90      0.90      0.90      8815
 weighted avg  0.94      0.94      0.94      8815
 
 Overall algorithm accuracy: 0.9372
\end{verbatim}

\hypertarget{gradient-boosting}{%
\paragraph{Gradient Boosting}\label{gradient-boosting}}

\begin{verbatim}
          precision    recall  f1-score   support
       1       0.98      0.92      0.95      3948
       2       0.85      0.87      0.86      3117
       3       0.70      0.67      0.69      1149
       4       0.60      0.82      0.70       471
       5       0.60      0.88      0.71       130
       
 micro avg     0.86      0.86      0.86      8815
 macro avg     0.75      0.83      0.78      8815
 weighted avg  0.87      0.86      0.87      8815
 
 Overall algorithm accuracy: 0.8635
\end{verbatim}

\hypertarget{the-best-lending-environment-simulator-model}{%
\subsubsection{The Best Lending Environment Simulator
Model}\label{the-best-lending-environment-simulator-model}}

The \textbf{Random Forest} algorithm has come up on top. This model
classifies all categories much better then the other two algorithms and
demonstrates a nice balance between the recall and precision metrics.
The Random forest algorithm is also the fastest to train.

\begin{longtable}[]{@{}llll@{}}
\toprule
Category & \textbf{RF f1-score} & SVM f1-score & GB
f1-score\tabularnewline
\midrule
\endhead
1 & \textbf{0.97} & 0.96 & 0.95\tabularnewline
2 & \textbf{0.93} & 0.91 & 0.86\tabularnewline
3 & \textbf{0.88} & 0.85 & 0.69\tabularnewline
4 & \textbf{0.85} & 0.83 & 0.70\tabularnewline
5 & \textbf{0.87} & 0.85 & 0.71\tabularnewline
Accuracy & \textbf{0.9372} & 0.9199 & 0.8635\tabularnewline
\bottomrule
\end{longtable}

\hypertarget{dimentionality-change}{%
\subsubsection{Dimentionality Change}\label{dimentionality-change}}

The winning algorithm performs quite spectacular. It employs the
\textbf{seven} top features we have identified in the \emph{Feature
Selection} section. Let's see how the input data dimentionality change
affects the model performance. Firstly we reduce the number of features
to \textbf{five}.

Top five features

\begin{longtable}[]{@{}ll@{}}
\toprule
Num & Feature\tabularnewline
\midrule
\endhead
24 & Do you receive your money back in time?\tabularnewline
26 & What's the most common use of the money you lend?\tabularnewline
22 & Do you include either interest or a lending fee when you
lend?\tabularnewline
18 & Over the past 3 months, how many times have you lent someone
money?\tabularnewline
20 & Who did you lend money to in the past 3 months?\tabularnewline
\bottomrule
\end{longtable}

Model Performance:

\begin{verbatim}
         precision    recall  f1-score   support
       1       0.94      0.91      0.92      3858
       2       0.84      0.74      0.79      3150
       3       0.58      0.73      0.65      1210
       4       0.55      0.71      0.62       464
       5       0.55      0.89      0.68       133

micro avg      0.81      0.81      0.81      8815
macro avg      0.69      0.80      0.73      8815
weighted avg   0.83      0.81      0.82      8815

Overall algorithm accuracy: 0.8635
\end{verbatim}

Evidently the dimentionality reduction caused the model performance
deteriorate greatly. Now let's increase the number of features to
\textbf{nine}.

Top nine features:

\begin{longtable}[]{@{}ll@{}}
\toprule
Num & Feature\tabularnewline
\midrule
\endhead
24 & Do you receive your money back in time?\tabularnewline
26 & What's the most common use of the money you lend?\tabularnewline
22 & Do you include either interest or a lending fee when you
lend?\tabularnewline
18 & Over the past 3 months, how many times have you lent someone
money?\tabularnewline
20 & Who did you lend money to in the past 3 months?\tabularnewline
23 & Do you request guarantees when you lend?\tabularnewline
19 & On average how much do you lend in general?\tabularnewline
16 & Country\tabularnewline
21 & When you lend money, when do you usually expect to get it
repaid?\tabularnewline
\bottomrule
\end{longtable}

Model Performance:

\begin{verbatim}
          precision    recall  f1-score   support

       1       0.99      0.98      0.98      3893
       2       0.96      0.96      0.96      3159
       3       0.89      0.92      0.91      1151
       4       0.85      0.87      0.86       482
       5       0.88      0.88      0.88       130

 micro avg     0.95      0.95      0.95      8815
 macro avg     0.91      0.92      0.92      8815
 weighted avg  0.96      0.95      0.95      8815
 
 Overall algorithm accuracy: 0.9547
\end{verbatim}

The dimentionality increase gave us a performance boost of almost
\textbf{2\%}. It might not seem much. Let see how performance of each
category has been affected.

\begin{longtable}[]{@{}lllll@{}}
\toprule
Category & 5 Features & 7 Features (base line) & \textbf{9 Features} &
Gain (\%)\tabularnewline
\midrule
\endhead
1 & 0.92 & 0.96 & \textbf{0.98} & 2\tabularnewline
2 & 0.79 & 0.93 & \textbf{0.96} & 3\tabularnewline
3 & 0.65 & 0.88 & \textbf{0.91} & 3\tabularnewline
4 & 0.62 & 0.85 & \textbf{0.86} & 1\tabularnewline
5 & 0.68 & 0.87 & \textbf{0.88} & 1\tabularnewline
\bottomrule
\end{longtable}

Evidently categories 2 (\emph{Poor}) and 3 (\emph{Fair}) have benefited
the most form the dimentionality increase. Ultimately it is up to the
business to decide if 3\% accuracy gain is worth the training time and
user interface complexity increase. KASI Insight representative has
opted for higher accuracy.

\hypertarget{hyper-parameter-tuning}{%
\subsubsection{Hyper-parameter Tuning}\label{hyper-parameter-tuning}}

Hyper-parameter tuning is usually the last step in effort to improve the
model performance. We will employ \emph{Grid Search} algorithm with
\textbf{three-fold cross validation} to identify the best model
parameters. The parameter grid look as follows:

\begin{longtable}[]{@{}ll@{}}
\toprule
Parameter & Values\tabularnewline
\midrule
\endhead
Number of Estimators & 200, 300, 400\tabularnewline
Minimum Sample Split & 5, 10, 20, 30, 40\tabularnewline
Maximum Features & `auto', `sqrt'\tabularnewline
Bootstrap: & True, False\tabularnewline
\bottomrule
\end{longtable}

The hyper-parameter tuning gave us another \textbf{0.5\%} performance
gain.

\hypertarget{final-simulator-model-stats}{%
\subsubsection{Final Simulator Model
Stats}\label{final-simulator-model-stats}}

\begin{itemize}
\tightlist
\item
  Number of features: \textbf{9}
\item
  Overall algorithm accuracy: \textbf{0.9594}
\end{itemize}

\begin{Schunk}
\begin{figure}[H]

{\centering \includegraphics[width=1\linewidth]{../../models/training/simulator_rf_tuned_large_matrix} 

}

\caption[Simulator Model Confusion Matrix]{Simulator Model Confusion Matrix}\label{fig:simulator_cm}
\end{figure}
\end{Schunk}

\begin{verbatim}
          precision    recall  f1-score   support

       1       0.99      0.99      0.99      3940
       2       0.96      0.96      0.96      3098
       3       0.91      0.90      0.90      1154
       4       0.83      0.88      0.86       471
       5       0.89      0.87      0.88       152

 micro avg     0.96      0.96      0.96      8815
 macro avg     0.92      0.92      0.92      8815
 weighted avg  0.96      0.96      0.96      8815
\end{verbatim}

Lastly we are going to review the model learning and validation curves.
As per figure \ref{fig:simulator_lc} the model was learning more about
the data as the training size grew. When the training size reached about
30,000 observations the validation curve converged with the training one
indicating that the further increase in the training set size will not
likely result in better model performance.

\begin{Schunk}
\begin{figure}[H]

{\centering \includegraphics[width=1\linewidth]{../../models/training/simulator_rf_tuned_large_curves} 

}

\caption[Simulator Model Learning Curves]{Simulator Model Learning Curves}\label{fig:simulator_lc}
\end{figure}
\end{Schunk}

\hypertarget{lender-evaluator}{%
\subsection{Lender Evaluator}\label{lender-evaluator}}

Without further due let's apply the same methodology to select
\emph{Lender Evaluator} classifier.

\hypertarget{svm-1}{%
\paragraph{SVM}\label{svm-1}}

\begin{verbatim}
          precision    recall  f1-score   support

       1       0.98      0.96      0.97      4053
       2       0.93      0.94      0.94      3249
       3       0.88      0.93      0.90      1272
       4       0.81      0.83      0.82       209
       5       0.75      0.66      0.70        32

micro avg      0.94      0.94      0.94      8815
macro avg      0.87      0.86      0.87      8815
weighted avg   0.94      0.94      0.94      8815

Overall algorithm accuracy: 0.9441
\end{verbatim}

\hypertarget{random-forest-1}{%
\paragraph{Random Forest}\label{random-forest-1}}

\begin{verbatim}
          precision    recall  f1-score   support

       1       0.99      0.98      0.99      4038
       2       0.97      0.98      0.97      3289
       3       0.95      0.95      0.95      1261
       4       0.89      0.82      0.85       197
       5       0.89      0.53      0.67        30

micro avg      0.97      0.97      0.97      8815
macro avg      0.94      0.85      0.89      8815
weighted avg   0.97      0.97      0.97      8815

Overall algorithm accuracy: 0.972
\end{verbatim}

\hypertarget{gradient-boosting-1}{%
\paragraph{Gradient Boosting}\label{gradient-boosting-1}}

\begin{verbatim}
          precision    recall  f1-score   support

       1       0.99      0.95      0.97      4066
       2       0.92      0.92      0.92      3261
       3       0.81      0.87      0.84      1218
       4       0.67      0.86      0.75       235
       5       0.68      0.80      0.74        35

micro avg      0.93      0.93      0.93      8815
macro avg      0.81      0.88      0.84      8815
weighted avg   0.93      0.93      0.93      8815

Overall algorithm accuracy: 0.925
\end{verbatim}

\hypertarget{the-best-lender-evaluator-model}{%
\subsubsection{The Best Lender Evaluator
Model}\label{the-best-lender-evaluator-model}}

Again the \textbf{Random Forest} algorithm proved to be the most
accurate. Though it does not identify category 5 observations as well as
the other two algorithm we hope that the input data dimentionality
increase and hyper-parameter tuning will rectify the problem.

\begin{longtable}[]{@{}llll@{}}
\toprule
Category & \textbf{RF f1-score} & SVM f1-score & GB
f1-score\tabularnewline
\midrule
\endhead
1 & \textbf{0.99} & 0.97 & 0.97\tabularnewline
2 & \textbf{0.97} & 0.94 & 0.92\tabularnewline
3 & \textbf{0.95} & 0.90 & 0.84\tabularnewline
4 & \textbf{0.85} & 0.82 & 0.75\tabularnewline
5 & \textbf{0.67} & 0.70 & 0.74\tabularnewline
Accuracy & \textbf{0.972} & 0.9441 & 0.925\tabularnewline
\bottomrule
\end{longtable}

\hypertarget{dimentionality-change-1}{%
\subsubsection{Dimentionality Change}\label{dimentionality-change-1}}

Let's evaluate how the number of features affect the accuracy of the
model. Just like in the case of the simulator model we begin with the
smaller feature set, namely seven.

Top seven features:

\begin{longtable}[]{@{}ll@{}}
\toprule
Num & Feature\tabularnewline
\midrule
\endhead
24 & Do you receive your money back in time?\tabularnewline
26 & What's the most common use of the money you lend?\tabularnewline
22 & Do you include either interest or a lending fee when you
lend?\tabularnewline
18 & Over the past 3 months, how many times have you lent someone
money?\tabularnewline
20 & Who did you lend money to in the past 3 months?\tabularnewline
23 & Do you request guarantees when you lend?\tabularnewline
19 & On average how much do you lend in general?\tabularnewline
\bottomrule
\end{longtable}

Model Performance:

\begin{verbatim}
      precision    recall  f1-score   support

       1       0.99      0.97      0.98      4005
       2       0.96      0.95      0.95      3330
       3       0.89      0.95      0.92      1252
       4       0.84      0.92      0.88       205
       5       0.84      0.70      0.76        23

micro avg      0.96      0.96      0.96      8815
macro avg      0.90      0.90      0.90      8815
weighted avg   0.96      0.96      0.96      8815 

Overall algorithm accuracy: 0.9585    
\end{verbatim}

Surprisingly the dimentionality reduction resulted in better model
performance!

Top ten features:

\begin{longtable}[]{@{}ll@{}}
\toprule
\begin{minipage}[b]{0.05\columnwidth}\raggedright
Num\strut
\end{minipage} & \begin{minipage}[b]{0.89\columnwidth}\raggedright
Feature\strut
\end{minipage}\tabularnewline
\midrule
\endhead
\begin{minipage}[t]{0.05\columnwidth}\raggedright
24\strut
\end{minipage} & \begin{minipage}[t]{0.89\columnwidth}\raggedright
Do you receive your money back in time?\strut
\end{minipage}\tabularnewline
\begin{minipage}[t]{0.05\columnwidth}\raggedright
18\strut
\end{minipage} & \begin{minipage}[t]{0.89\columnwidth}\raggedright
Over the past 3 months, how many times have you lent someone
money?\strut
\end{minipage}\tabularnewline
\begin{minipage}[t]{0.05\columnwidth}\raggedright
22\strut
\end{minipage} & \begin{minipage}[t]{0.89\columnwidth}\raggedright
Do you include either interest or a lending fee when you lend?\strut
\end{minipage}\tabularnewline
\begin{minipage}[t]{0.05\columnwidth}\raggedright
23\strut
\end{minipage} & \begin{minipage}[t]{0.89\columnwidth}\raggedright
Do you request guarantees when you lend?\strut
\end{minipage}\tabularnewline
\begin{minipage}[t]{0.05\columnwidth}\raggedright
20\strut
\end{minipage} & \begin{minipage}[t]{0.89\columnwidth}\raggedright
Who did you lend money to in the past 3 months?\strut
\end{minipage}\tabularnewline
\begin{minipage}[t]{0.05\columnwidth}\raggedright
26\strut
\end{minipage} & \begin{minipage}[t]{0.89\columnwidth}\raggedright
What's the most common use of the money you lend?\strut
\end{minipage}\tabularnewline
\begin{minipage}[t]{0.05\columnwidth}\raggedright
19\strut
\end{minipage} & \begin{minipage}[t]{0.89\columnwidth}\raggedright
On average how much do you lend in general?\strut
\end{minipage}\tabularnewline
\begin{minipage}[t]{0.05\columnwidth}\raggedright
25\strut
\end{minipage} & \begin{minipage}[t]{0.89\columnwidth}\raggedright
Assuming that you have lent money at least ten times, how often would
you get your money repaid?\strut
\end{minipage}\tabularnewline
\begin{minipage}[t]{0.05\columnwidth}\raggedright
16\strut
\end{minipage} & \begin{minipage}[t]{0.89\columnwidth}\raggedright
Country\strut
\end{minipage}\tabularnewline
\begin{minipage}[t]{0.05\columnwidth}\raggedright
21\strut
\end{minipage} & \begin{minipage}[t]{0.89\columnwidth}\raggedright
When you lend money, when do you usually expect to get it repaid?\strut
\end{minipage}\tabularnewline
\bottomrule
\end{longtable}

Model Performance:

\begin{verbatim}
          precision    recall  f1-score   support

       1       0.99      0.99      0.99      4058
       2       0.97      0.98      0.97      3216
       3       0.94      0.96      0.95      1273
       4       0.92      0.85      0.88       229
       5       0.97      0.74      0.84        39

micro avg      0.98      0.98      0.98      8815
macro avg      0.96      0.90      0.93      8815
weighted avg   0.98      0.98      0.98      8815

Overall algorithm accuracy: 0.975
\end{verbatim}

The dimentionality increase yields the best performance. The category 5
identification has been improved the most. The top ten feature set is an
ultimate winner.

\begin{longtable}[]{@{}lllll@{}}
\toprule
Category & 7 Features** & 9 Features (base-line) & \textbf{10 Features}
& Gain (\%)\tabularnewline
\midrule
\endhead
1 & 0.98 & 0.99 & \textbf{0.99} & 0\tabularnewline
2 & 0.95 & 0.97 & \textbf{0.97} & 0\tabularnewline
3 & 0.92 & 0.95 & \textbf{0.95} & 0\tabularnewline
4 & 0.88 & 0.85 & \textbf{0.88} & 3\tabularnewline
5 & 0.76 & 0.67 & \textbf{0.84} & 17\tabularnewline
\bottomrule
\end{longtable}

\hypertarget{hyper-parameter-tuning-1}{%
\subsubsection{Hyper-parameter Tuning}\label{hyper-parameter-tuning-1}}

The selected model demonstrates quite spectacular accuracy. We could
potentially improve the identification of the category \# 5. We are
going to apply Grid Search algorithm again trying to find the optimal
hyper-parameters.

Parameter grid:

\begin{longtable}[]{@{}ll@{}}
\toprule
Parameter & Values\tabularnewline
\midrule
\endhead
Number of Estimators & 200, 300, 400\tabularnewline
Minimum Sample Split & 5, 10, 20, 30, 40\tabularnewline
Maximum Features & `auto', `sqrt'\tabularnewline
Bootstrap: & True, False\tabularnewline
\bottomrule
\end{longtable}

The hyper-parameter tuning has demonstrated similar to the default
parameter performance.

\hypertarget{final-evaluator-model-stats}{%
\subsubsection{Final Evaluator Model
Stats}\label{final-evaluator-model-stats}}

\begin{itemize}
\tightlist
\item
  Number of features: \textbf{10}
\item
  Overall algorithm accuracy: \textbf{0.9756}
\end{itemize}

\begin{Schunk}
\begin{figure}[H]

{\centering \includegraphics[width=1\linewidth]{../../models/training/evaluator_rf_tuned_large_matrix} 

}

\caption[Evaluator Model Confusion Matrix]{Evaluator Model Confusion Matrix}\label{fig:evaluator_cm}
\end{figure}
\end{Schunk}

\begin{verbatim}
       precision    recall  f1-score   support

       1       0.99      0.99      0.99      4110
       2       0.98      0.98      0.98      3237
       3       0.94      0.96      0.95      1217
       4       0.90      0.82      0.86       214
       5       0.96      0.73      0.83        37

micro avg      0.98      0.98      0.98      8815
macro avg      0.95      0.90      0.92      8815
weighted avg   0.98      0.98      0.98      8815
\end{verbatim}

Lastly we are going to review the model learning and validation curves.
As per figure \ref{fig:evaluator_lc} the model was learning more about
the data as the training size grew. When the training size reached about
30,000 observations the validation curve converged with the training one
indicating that the further increase in the training set size will not
likely result in better model performance.

\begin{Schunk}
\begin{figure}[H]

{\centering \includegraphics[width=1\linewidth]{../../models/training/evaluator_rf_tuned_large_curves} 

}

\caption[Eval Model Learning Curves]{Eval Model Learning Curves}\label{fig:evaluator_lc}
\end{figure}
\end{Schunk}

\hypertarget{model-deployment}{%
\section{Model Deployment}\label{model-deployment}}

\hypertarget{architecture}{%
\subsection{Architecture}\label{architecture}}

\hypertarget{docker}{%
\subsection{Docker}\label{docker}}

\hypertarget{conclusion}{%
\section{Conclusion}\label{conclusion}}

\bibliography{RJreferences}

\hypertarget{note-from-the-authors}{%
\section{Note from the Authors}\label{note-from-the-authors}}

This file was generated using
\href{https://github.com/rstudio/rticles}{\emph{The R Journal} style
article template}, additional information on how to prepare articles for
submission is here -
\href{https://journal.r-project.org/share/author-guide.pdf}{Instructions
for Authors}. The article itself is an executable R Markdown file that
could be
\href{https://github.com/ivbsoftware/big-data-final-2/blob/master/docs/R_Journal/big-data-final-2/}{downloaded
from Github} with all the necessary artifacts.


\address{%
Vadim Spirkov\\
York University School of Continuing Studies\\
\\
}


\address{%
Murlidhar Loka\\
York University School of Continuing Studies\\
\\
}


