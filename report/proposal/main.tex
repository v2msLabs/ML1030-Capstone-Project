% !TeX root = RJwrapper.tex
\title{Lending Environment Simulation Model \& Lender Evaluation Tool}
\author{by Vadim Spirkov, Murlidhar Loka}

\maketitle

\abstract{%
In 2013, Africa was the world's fastest-growing continent at 5.6\% a
year, and GDP is expected to rise by an average of over 6\% a year
between 2013 and 2023. In 2017, the African Development Bank reported
Africa to be the world's second-fastest growing economy, and estimates
that average growth will rebound to 3.4\% in 2017, while growth is
expected to increase by 4.3\% in 2018. The World Bank expects that most
African countries will reach ``middle income'' status (defined as at
least US\$1,000 per person a year) by 2025 if current growth rates
continue.(Ref: \cite{wiki}) With growth of the income African and
international financial institutions are looking for the opportunities
to extend their services to the new geographic areas.
}

% Any extra LaTeX you need in the preamble

\hypertarget{introduction}{%
\section{Introduction}\label{introduction}}

According to McKinsey, Africa has the third largest population of
unbanked adults (326 millions) representing 80\% of the adult's
population. According to banking executives in Africa, assessing
consumer credit risk is one of the main challenges the banks face, which
results in the slowing the access of general public to the credit
resources.

KASI Insights is an award-winning consumer and market intelligence firm
that provides biometric research at scale in Africa. KASI offers the
first crowd sourced credit model. KASI Insights engages, questions and
learns in real time from consumers and market participants in over 10
markets in Africa to uncover what they really feel about the products
and services the companies consider to offer in the continent before
they launch them. (Ref:\cite{kasi})

\hypertarget{background}{%
\section{Background}\label{background}}

In Africa money lending between the individuals is very common. This
practice in fact is a substitution to the small loan franchises, which
are wide-spread in Western countries. One of the surveys KASI conducts
for years targets the individuals who lend money to other members of the
community on regular basis. The firm leverages the wisdom of the crowd
to evaluate lending activities within a community over time to compute a
community-based score that can be translated into an individual score.
Thus KASI's surveys provide risk profile of the population in a given
area through the eyes of the lender.

Having many conversations with the KASI Insights CEO, we have concluded
that, if properly used, these surveys could be a valuable intel for the
banks and other financial institution to evaluate a possibility of
establishing branches in suitable communities to conduct small loan
business.

The data collected by KASI Insight could also be used by the banks and
small lenders alike to evaluate the lending environment in African
countries/communities simulating various lending criteria.

In this context we hope that applying Machine learning techniques and
the latest advances in the software development we will put the
collected data to use and help KASI Insights increase the offering to
the interested parties, thus improving the company's position in the
field of economic and marketing research in Africa.

\hypertarget{problem-statement}{%
\section{Problem Statement}\label{problem-statement}}

This project has two objectives.

\hypertarget{design-and-implement-lending-environment-simulation-model}{%
\subsection{Design and Implement Lending Environment Simulation
Model}\label{design-and-implement-lending-environment-simulation-model}}

The first objective is to help KASI Insights to provide a valuable tool
to the financial institutions and the small lenders in Africa, which
will be used to \textbf{predict a credit score class of the target group
of people in a given area manipulating multiple lending criteria}.

\hypertarget{approach}{%
\subsubsection{Approach}\label{approach}}

The brain of the tool is going to be a \textbf{multiclass classification
model}. We will be employing a credit score formula provided by KASI
Insights and lender's survey data to label each observation. KASI
Insights has come up with the following credit score classes:

\begin{itemize}
\tightlist
\item
  \textbf{Very Poor} - the credit score less than \textbf{350}
\item
  \textbf{Poor} - the credit score is between \textbf{350 and 550}
\item
  \textbf{Fair} - the credit score is between \textbf{550 and 650}
\item
  \textbf{Good} - the credit score lies between \textbf{650 and 750}
\item
  \textbf{Very Good} - the credit score is higher tan \textbf{750}
\end{itemize}

We also will provide a Web-based user interface to communicate with the
model. We plan to develop the solution employing \emph{Python} and
\emph{Angular} JavaScript framework. The model will be hosted in
\emph{Flask} Web application server. The code and the supporting
software will be shipped to the client as a \emph{Docker} package for
the deployment on \textbf{AWS} cloud.

\hypertarget{lender-evaluation-tool}{%
\subsection{Lender Evaluation Tool}\label{lender-evaluation-tool}}

Lender evaluation tool will provide means to \textbf{estimate the
business savviness of an individual who lends money on regular basis}.
It is meant to be used by the banks and other financial institutions
which are interested in having a proxy in a given community to conduct
small loan business in their behalf.

KASI insight has committed to formulate an algorithm that calculates the
worthiness of a lender as a business partner in the money lending
context. We will be employing the demographic data and the lending
habits of the lender contained in the survey data and the lender
worthiness score to label all observations.

At the moment of writing we did not know if KASI Insights wanted to
predict the lender worthiness score as an absolute number or use a few
classes that would define how good/bad the lender as a business partner
is. Depending on the final decision we will train either a
\textbf{regression} or \textbf{multiclass classification} model.

Similarly to the first objective deliverable the final product will be a
Web-based solution implemented using \emph{Python} and \emph{Angular}.
The code and supplementary software will be \emph{dockerized} for easy
roll-out on a cloud platform.

\hypertarget{dataset-description}{%
\section{Dataset Description}\label{dataset-description}}

As it has been mentioned previously we will be using KASI lender survey
data for the project. The data has been collected surveying people from
seven African countries over a course of the last three years. The data
is maintained in the Excel spreadsheets in English and French. The
survey comprises 38 multi-choice questions. There are almost
\textbf{30,000} observations. The table submitted below lists the survey
questions:

\begin{longtable}[]{@{}ll@{}}
\toprule
\begin{minipage}[b]{0.05\columnwidth}\raggedright
ID\strut
\end{minipage} & \begin{minipage}[b]{0.89\columnwidth}\raggedright
Question/Column\strut
\end{minipage}\tabularnewline
\midrule
\endhead
\begin{minipage}[t]{0.05\columnwidth}\raggedright
0\strut
\end{minipage} & \begin{minipage}[t]{0.89\columnwidth}\raggedright
Timestamp\strut
\end{minipage}\tabularnewline
\begin{minipage}[t]{0.05\columnwidth}\raggedright
1\strut
\end{minipage} & \begin{minipage}[t]{0.89\columnwidth}\raggedright
Location ID\strut
\end{minipage}\tabularnewline
\begin{minipage}[t]{0.05\columnwidth}\raggedright
2\strut
\end{minipage} & \begin{minipage}[t]{0.89\columnwidth}\raggedright
Has it become more difficult or easier to find a job in your city?\strut
\end{minipage}\tabularnewline
\begin{minipage}[t]{0.05\columnwidth}\raggedright
3\strut
\end{minipage} & \begin{minipage}[t]{0.89\columnwidth}\raggedright
Is this a good time for people to make a large purchase such as
furniture or electrical appliances, given the economic climate?\strut
\end{minipage}\tabularnewline
\begin{minipage}[t]{0.05\columnwidth}\raggedright
4\strut
\end{minipage} & \begin{minipage}[t]{0.89\columnwidth}\raggedright
Compared to the last 6 months, are you able to spend (more, the same or
less) money on large purchases over the next 6 months?\strut
\end{minipage}\tabularnewline
\begin{minipage}[t]{0.05\columnwidth}\raggedright
5\strut
\end{minipage} & \begin{minipage}[t]{0.89\columnwidth}\raggedright
Will you be able to meet your regular expenses over the next 6
months?\strut
\end{minipage}\tabularnewline
\begin{minipage}[t]{0.05\columnwidth}\raggedright
6\strut
\end{minipage} & \begin{minipage}[t]{0.89\columnwidth}\raggedright
How do you expect your household's income to change over the next 6
months?\strut
\end{minipage}\tabularnewline
\begin{minipage}[t]{0.05\columnwidth}\raggedright
7\strut
\end{minipage} & \begin{minipage}[t]{0.89\columnwidth}\raggedright
How do you expect general economic conditions in your city to change
over the next 6 months?\strut
\end{minipage}\tabularnewline
\begin{minipage}[t]{0.05\columnwidth}\raggedright
8\strut
\end{minipage} & \begin{minipage}[t]{0.89\columnwidth}\raggedright
How do you expect general economic conditions in your country to change
over the next 6 months?\strut
\end{minipage}\tabularnewline
\begin{minipage}[t]{0.05\columnwidth}\raggedright
9\strut
\end{minipage} & \begin{minipage}[t]{0.89\columnwidth}\raggedright
Gender\strut
\end{minipage}\tabularnewline
\begin{minipage}[t]{0.05\columnwidth}\raggedright
10\strut
\end{minipage} & \begin{minipage}[t]{0.89\columnwidth}\raggedright
Marital status\strut
\end{minipage}\tabularnewline
\begin{minipage}[t]{0.05\columnwidth}\raggedright
11\strut
\end{minipage} & \begin{minipage}[t]{0.89\columnwidth}\raggedright
Age\strut
\end{minipage}\tabularnewline
\begin{minipage}[t]{0.05\columnwidth}\raggedright
12\strut
\end{minipage} & \begin{minipage}[t]{0.89\columnwidth}\raggedright
What's your highest level of education?\strut
\end{minipage}\tabularnewline
\begin{minipage}[t]{0.05\columnwidth}\raggedright
13\strut
\end{minipage} & \begin{minipage}[t]{0.89\columnwidth}\raggedright
Occupation\strut
\end{minipage}\tabularnewline
\begin{minipage}[t]{0.05\columnwidth}\raggedright
14\strut
\end{minipage} & \begin{minipage}[t]{0.89\columnwidth}\raggedright
If you are a student, what level are you currently studying?\strut
\end{minipage}\tabularnewline
\begin{minipage}[t]{0.05\columnwidth}\raggedright
15\strut
\end{minipage} & \begin{minipage}[t]{0.89\columnwidth}\raggedright
Race/Ethnicity\strut
\end{minipage}\tabularnewline
\begin{minipage}[t]{0.05\columnwidth}\raggedright
16\strut
\end{minipage} & \begin{minipage}[t]{0.89\columnwidth}\raggedright
Country\strut
\end{minipage}\tabularnewline
\begin{minipage}[t]{0.05\columnwidth}\raggedright
17\strut
\end{minipage} & \begin{minipage}[t]{0.89\columnwidth}\raggedright
What is the name of the neighborhood where you live?\strut
\end{minipage}\tabularnewline
\begin{minipage}[t]{0.05\columnwidth}\raggedright
18\strut
\end{minipage} & \begin{minipage}[t]{0.89\columnwidth}\raggedright
Over the past 3 months, how many times have you lent someone
money?\strut
\end{minipage}\tabularnewline
\begin{minipage}[t]{0.05\columnwidth}\raggedright
19\strut
\end{minipage} & \begin{minipage}[t]{0.89\columnwidth}\raggedright
On average how much do you lend in general?\strut
\end{minipage}\tabularnewline
\begin{minipage}[t]{0.05\columnwidth}\raggedright
20\strut
\end{minipage} & \begin{minipage}[t]{0.89\columnwidth}\raggedright
Who did you lend money to in the past 3 months?\strut
\end{minipage}\tabularnewline
\begin{minipage}[t]{0.05\columnwidth}\raggedright
21\strut
\end{minipage} & \begin{minipage}[t]{0.89\columnwidth}\raggedright
When you lend money, when do you usually expect to get it repaid?\strut
\end{minipage}\tabularnewline
\begin{minipage}[t]{0.05\columnwidth}\raggedright
22\strut
\end{minipage} & \begin{minipage}[t]{0.89\columnwidth}\raggedright
Do you include either interest or a lending fee when you lend?\strut
\end{minipage}\tabularnewline
\begin{minipage}[t]{0.05\columnwidth}\raggedright
23\strut
\end{minipage} & \begin{minipage}[t]{0.89\columnwidth}\raggedright
Do you request guarantees when you lend?\strut
\end{minipage}\tabularnewline
\begin{minipage}[t]{0.05\columnwidth}\raggedright
24\strut
\end{minipage} & \begin{minipage}[t]{0.89\columnwidth}\raggedright
Do you receive your money back in time?\strut
\end{minipage}\tabularnewline
\begin{minipage}[t]{0.05\columnwidth}\raggedright
25\strut
\end{minipage} & \begin{minipage}[t]{0.89\columnwidth}\raggedright
Assuming that you have lent money at least ten times, how often would
you get your money repaid?\strut
\end{minipage}\tabularnewline
\begin{minipage}[t]{0.05\columnwidth}\raggedright
26\strut
\end{minipage} & \begin{minipage}[t]{0.89\columnwidth}\raggedright
What's the most common use of the money you lend?\strut
\end{minipage}\tabularnewline
\begin{minipage}[t]{0.05\columnwidth}\raggedright
27\strut
\end{minipage} & \begin{minipage}[t]{0.89\columnwidth}\raggedright
Have you ever applied for a bank loan?\strut
\end{minipage}\tabularnewline
\begin{minipage}[t]{0.05\columnwidth}\raggedright
28\strut
\end{minipage} & \begin{minipage}[t]{0.89\columnwidth}\raggedright
Are you a tontine / lending club member?\strut
\end{minipage}\tabularnewline
\begin{minipage}[t]{0.05\columnwidth}\raggedright
29\strut
\end{minipage} & \begin{minipage}[t]{0.89\columnwidth}\raggedright
What is the most convenient way to get a loan?\strut
\end{minipage}\tabularnewline
\begin{minipage}[t]{0.05\columnwidth}\raggedright
30\strut
\end{minipage} & \begin{minipage}[t]{0.89\columnwidth}\raggedright
To what extent do you agree with the following sentences {[}Access to
credit is essential for me to achieve financial freedom{]}\strut
\end{minipage}\tabularnewline
\begin{minipage}[t]{0.05\columnwidth}\raggedright
31\strut
\end{minipage} & \begin{minipage}[t]{0.89\columnwidth}\raggedright
To what extent do you agree with the following sentences {[}Credit is
beneficial only if you have discipline{]}\strut
\end{minipage}\tabularnewline
\begin{minipage}[t]{0.05\columnwidth}\raggedright
32\strut
\end{minipage} & \begin{minipage}[t]{0.89\columnwidth}\raggedright
To what extent do you agree with the following sentences {[}I would like
to have more credit management training{]}\strut
\end{minipage}\tabularnewline
\begin{minipage}[t]{0.05\columnwidth}\raggedright
33\strut
\end{minipage} & \begin{minipage}[t]{0.89\columnwidth}\raggedright
What type of loans are you currently paying of?\strut
\end{minipage}\tabularnewline
\begin{minipage}[t]{0.05\columnwidth}\raggedright
34\strut
\end{minipage} & \begin{minipage}[t]{0.89\columnwidth}\raggedright
Do you have a credit score?\strut
\end{minipage}\tabularnewline
\begin{minipage}[t]{0.05\columnwidth}\raggedright
35\strut
\end{minipage} & \begin{minipage}[t]{0.89\columnwidth}\raggedright
Do you have a credit card?\strut
\end{minipage}\tabularnewline
\begin{minipage}[t]{0.05\columnwidth}\raggedright
36\strut
\end{minipage} & \begin{minipage}[t]{0.89\columnwidth}\raggedright
On average, how much of your total household monthly income do you spend
paying off debt each month?\strut
\end{minipage}\tabularnewline
\begin{minipage}[t]{0.05\columnwidth}\raggedright
37\strut
\end{minipage} & \begin{minipage}[t]{0.89\columnwidth}\raggedright
If you wanted to take a loan to start a business, how much would you
need?\strut
\end{minipage}\tabularnewline
\bottomrule
\end{longtable}

Since the survey has a multi-choice format the data could be easily
categorized prior to fitting the model. As discussed the labels for each
observation are:

\begin{itemize}
\item
  Credit score category based on the credit score. This label will be
  employed for training of the \textbf{Lending Environment Simulation
  Model}
\item
  Lender business worthiness score or a category (TBD). This label will
  be used to train the \textbf{Lender Evaluation Model}
\end{itemize}

\hypertarget{data-pollution-challenge-solution}{%
\subsection{Data Pollution Challenge \&
Solution}\label{data-pollution-challenge-solution}}

The survey has been modified a few time since its inception. This lead
to data pollution. In the latest version of survey each question offers
a few possible answers, 3 - 4 on average, some questions have up to nine
possible answers. In realty the preliminary data analysis shows that
some questions have dozens, in extreme cases even up to 1000 possible
answers\ldots{}

\hypertarget{solution}{%
\subsubsection{Solution}\label{solution}}

To rectify the problem of the data pollution we suggest and have agreed
with the business to use the latest version of survey to categorize the
answers. Those answers that do not fall into any of the available
categories will be categorized as \emph{Others}. Considering the fact
the the survey has not been change much for the last couple years we
expect that the majority of the data will be correctly classified. On
top of it will apply sophisticated algorithm that try to categorize even
polluted answers.

Considering racial nature of the question \#15 we have reached an
agreement with the business to exclude the question from the process.

\hypertarget{data-balance}{%
\subsection{Data balance}\label{data-balance}}

At the moment of writing we did not know if the dataset was balanced. If
it is not we would consider the data upsamling techniques.

\hypertarget{ml-solution}{%
\section{ML Solution}\label{ml-solution}}

To meet the project objectives we propose the following pipeline.

\hypertarget{data-preprocessing}{%
\subsection{Data Preprocessing}\label{data-preprocessing}}

\begin{enumerate}
\def\labelenumi{\arabic{enumi}.}
\tightlist
\item
  Convert data from Excel format to *.csv format and encode it in UTF-8
  encoding. Save the data into two files: one file will contain the data
  in English and the other in French.
\item
  Develop a script to clean and categorize the data using the latest
  KASI survey to identify the categories correctly
\item
  Using the formulas provided by KASI Insights label each observations.
  There will be two labels: one for the credit category class and the
  other for the lender business worthiness. Save the clean and labeled
  data into a *.csv file.
\item
  Document data preprocessing steps so our business partner could
  reproduce it when the model re-training is required.
\end{enumerate}

\hypertarget{model-selection-and-training}{%
\subsection{Model Selection and
Training}\label{model-selection-and-training}}

\begin{enumerate}
\def\labelenumi{\arabic{enumi}.}
\tightlist
\item
  For each proposed model we will evaluate three algorithms:
\end{enumerate}

\begin{itemize}
\tightlist
\item
  \textbf{Support Vector Machine} (SVM). The greatest strength of SVM is
  that it has multiple Kernel implementations, that can be tuned to
  explain multi-dimensional space with high accuracy
\item
  \textbf{Random Forest} (RF). Random forest belongs to the class of
  ensemble models. It has many hyperparameters that could be tuned to
  achieve high accuracy. The random forest algorithm is not demanding in
  terms of the data preparation, which makes it the first choice in many
  real-life scenarios
\item
  \textbf{Gradient Boosting Machine} (GBM). GBM is an ensemble model as
  well. It uses the concept of trees just like the RF model does but
  applies it differently. GBT builds the trees one at a time, where each
  new tree helps to correct errors made by previously trained tree. the
  GBM.
\end{itemize}

If time permits we will attempt to evaluate a Neural Network in addition
to the aforementioned models.

The listed above models are good for classification and regression
tasks. Thus if KASI insights decides to employ regression approach for
the second project we are covered.

\begin{enumerate}
\def\labelenumi{\arabic{enumi}.}
\setcounter{enumi}{1}
\item
  Plot and interpret the model learning curves to detect possible
  problems such as overfitting, luck of training data, etc.
\item
  As an evaluation criteria we will employ the multiclass confusion
  matrix and averaged micro and macro \textbf{F1 scores}. The multiclass
  confusion matrix will clearly tell how well each algorithm detects the
  categories. We believe that the F1 score is ideal metric for the model
  evaluation because it provides the \textbf{balanced} value of
  accuracy.
\item
  Pick the winning model taking into account the following criteria:
\end{enumerate}

\begin{itemize}
\tightlist
\item
  Model prediction power (the higher F1 score the better)
\item
  Model performance. Considering the fact that the model will be used
  on-line we shall no underestimate this factor; the faster the model
  the better.
\item
  Training time. For time being, if required, KASI Insight will be
  re-training the model using PC. Thus re-training the model shall not
  be taking days\ldots{}
\end{itemize}

\hypertarget{project-plan}{%
\section{Project Plan}\label{project-plan}}

We have eight weeks to make a dream a reality.

\hypertarget{week-1}{%
\subsection{Week 1}\label{week-1}}

Understand the business domain and the data. Discuss the goal the
business partner wants to achieve and evaluate the feasibility of the
task. Do gap analysis. If the goal the client has in mind is not
achievable due to either luck of data, time constraints or
misunderstanding of the ML capabilities offer alternative solution,
which the client could benefit from. Get all missing pieces of the
information from the business partner. Finish and submit project
proposal.

\hypertarget{week-2}{%
\subsection{Week 2}\label{week-2}}

Clean the data. Do data exploration. Address possible data issues such
as unbalanced data. Document the data exploration findings and outcomes.

\hypertarget{weeks-3---4}{%
\subsection{Weeks 3 - 4}\label{weeks-3---4}}

Train and evaluate the models. Pick the winning model using the
evaluation criteria discussed earlier. Document the process. Submit the
first milestone assignment.

\hypertarget{weeks-5---7}{%
\subsection{Weeks 5 - 7}\label{weeks-5---7}}

Develop turnkey solution:

\begin{itemize}
\tightlist
\item
  Develop Web-client employing \emph{Angular} JavaScript framework.
\item
  Deploy the model to \emph{Flask} Web application service.
\item
  Test the solution end-to-end locally.
\item
  \emph{Dockerize} the code.
\item
  Deploy to AWS cloud.
\end{itemize}

Submit \textbf{Project Solution} assignment.

\hypertarget{week-8}{%
\subsection{Week 8}\label{week-8}}

Address possible issues with the code. Polish the project report and the
presentation. Educate the client on how to use the product, re-train the
model. Advise the client what could be improved and what possible
challenges the client might have in future. Do final submission. Ship
all project artifacts to the client.

\bibliography{RJreferences}

\hypertarget{note-from-the-authors}{%
\section{Note from the Authors}\label{note-from-the-authors}}

This file was generated using
\href{https://github.com/rstudio/rticles}{\emph{The R Journal} style
article template}, additional information on how to prepare articles for
submission is here -
\href{https://journal.r-project.org/share/author-guide.pdf}{Instructions
for Authors}. The article itself is an executable R Markdown file that
could be
\href{https://github.com/ivbsoftware/big-data-final-2/blob/master/docs/R_Journal/big-data-final-2/}{downloaded
from Github} with all the necessary artifacts.


\address{%
Vadim Spirkov\\
York University School of Continuing Studies\\
\\
}


\address{%
Murlidhar Loka\\
York University School of Continuing Studies\\
\\
}


